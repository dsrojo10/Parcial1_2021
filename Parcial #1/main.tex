\documentclass{article}
\usepackage[utf8]{inputenc}
\usepackage[spanish]{babel}
\usepackage{listings}
\usepackage{graphicx}
\graphicspath{ {images/} }
\usepackage{cite}

\begin{document}

\begin{titlepage}
    \begin{center}
        \vspace*{1cm}
            
        \Huge
        \textbf{PARCIAL \#1}
            
        \vspace{0.5cm}
        \LARGE
        Informa2 S.A.S.
            
        \vspace{1.5cm}
            
        \textbf{David Santiago Rojo Castrillon}
            
        \vfill
            
        \vspace{0.8cm}
            
        \Large
        Departamento de Ingeniería Electrónica y Telecomunicaciones\\
        Universidad de Antioquia\\
        Medellín\\
        Abril de 2021
            
    \end{center}
\end{titlepage}

\tableofcontents
\newpage
\section{Sección introductoria}\label{intro}
El principal objetivo de esta actividad es poner en práctica los conocimientos aprendidos en informática II y en cursos previos de programación o circuitos eléctricos, se pondrán a prueba las habilidades en cuanto a la resolución de problemas. Es una actividad que además de combinar la programación y el montaje de circuitos con Arduino a la vez permite afrontar problemas del mundo laboral.

\section{Análisis del problema} \label{contenido}

\section{Tareas definidas} \label{contenido}
subsection{Investigación}
Para llevar a cabo esta actividad son necesarios algunos conceptos adicionales, ya que usar pocos puertos y conectar varios integrados en serie son conceptos desconocidos por ahora, es coherente hacer una investigación sobre la documentación del integrado y consultar mas sobre los puertos digitales de un Arduino.

\subsection{Montaje circuito}
El montaje del circuito es una de las principales tareas a afrontar, se debe hacer el arreglo de 64 leds, con sus respectivas resistencias, para controlar esta cantidad de leds a través el Arduino haremos uso del circuito integrado 74HC595. En este caso se usará un C.I. por cada fila (en total 8) haciendo uso de 3 pines digitales. 


\subsection{Codificacion tareas}

\section{Problemas presentados} \label{contenido}

\section{Consideraciones} \label{contenido}



\section{Inclusión de imágenes} \label{imagenes}

En la Figura (\ref{fig:cpplogo}), se presenta el logo de C++ contenido en la carpeta images.

\begin{figure}[h]
\includegraphics[width=4cm]{cpplogo.png}
\centering
\caption{Logo de C++}
\label{fig:cpplogo}
\end{figure}

Las secciones (\ref{intro}), (\ref{contenido}) y (\ref{imagenes}) dependen del estilo del documento.

\bibliographystyle{IEEEtran}
\bibliography{references}

\end{document}
